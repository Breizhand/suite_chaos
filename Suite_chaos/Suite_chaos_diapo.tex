\documentclass{beamer}

\usepackage[frenchb]{babel}
\usepackage[T1]{fontenc}
\usepackage[utf8]{inputenc}

\usetheme{Warsaw}
\setbeamertemplate{navigation symbols}{%
\insertslidenavigationsymbol % Icône slide
\insertframenavigationsymbol % Icône frame
\insertsubsectionnavigationsymbol % Icône sous section
\insertsectionnavigationsymbol % Icône section
\insertdocnavigationsymbol % Icône docnavigation
\insertbackfindforwardnavigationsymbol % Icône backfindforward
}


\title{Suite et Chaos}
\author{Chcouropat Youri - Tartu Cédric}
\institute{Brest Open Campus - EPSI}
\date{20 avril 2017}

\begin{document}

\begin{frame}
\titlepage
\end{frame}

\begin{frame}
\frametitle{Présentation}
Ce document est la présentation du projet de mathématiques entre les suites et le chaos.
Nous allons découvrir quels liens se cachent entre ces deux objets mathématiques à priori totalement indépendants l'un de l'autre.
\end{frame}

\begin{frame}
\frametitle{Sommaire}
  \tableofcontents
\end{frame}

\begin{frame}
\frametitle{Les Suites}
\end{frame}

\begin{frame}
\frametitle{Le Chaos}
\end{frame}

\begin{frame}
\frametitle{Les Fractales}
\end{frame}

\begin{frame}
\frametitle{Lien Chaos et Fractales}
\end{frame}

\begin{frame}
\frametitle{Le Nombre d'Or}
\end{frame}

\begin{frame}
\frametitle{Conclusion}
\end{frame}



\end{document}